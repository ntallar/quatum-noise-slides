%% This Beamer template is based on the one found here: https://github.com/sanhacheong/stanford-beamer-presentation, and edited to be used for Stanford ARM Lab

\documentclass[10pt]{beamer}
%\mode<presentation>{}

\usepackage{media9}
\usepackage{amssymb,amsmath,amsthm,enumerate}
\usepackage[utf8]{inputenc}
\usepackage{array}
\usepackage[parfill]{parskip}
\usepackage{graphicx}
\usepackage{caption}
\usepackage{subcaption}
\usepackage{bm}
\usepackage{amsfonts,amscd}
\usepackage[]{units}
\usepackage{listings}
\usepackage{multicol}
\usepackage{multirow}
\usepackage{tcolorbox}
\usepackage{physics}

% Packages for writing in spanish
\usepackage[utf8]{inputenc}
\usepackage[spanish]{babel}
\usepackage[T1]{fontenc}

% Packages for quantum circuits
\usepackage{tikz}
\usetikzlibrary{quantikz}

% Package for subfigures
\usepackage{subcaption}

% For using gather for equations
\usepackage{amsmath}

% Enable colored hyperlinks
\hypersetup{colorlinks=true}

% The following three lines are for crossmarks & checkmarks
\usepackage{pifont}% http://ctan.org/pkg/pifont
\newcommand{\cmark}{\ding{51}}%
\newcommand{\xmark}{\ding{55}}%

% Numbered captions of tables, pictures, etc.
\setbeamertemplate{caption}[numbered]

%\usepackage[superscript,biblabel]{cite}
\usepackage{algorithm2e}
\renewcommand{\thealgocf}{}

% Bibliography settings
\usepackage[style=ieee]{biblatex}
\setbeamertemplate{bibliography item}{\insertbiblabel}
\addbibresource{references.bib}

% Glossary entries
\usepackage[acronym]{glossaries}
\newacronym{ML}{ML}{machine learning}
\newacronym{HRI}{HRI}{human-robot interactions}
\newacronym{RNN}{RNN}{Recurrent Neural Network}
\newacronym{LSTM}{LSTM}{Long Short-Term Memory}


\theoremstyle{remark}
\newtheorem*{remark}{Remark}
\theoremstyle{definition}

\newcommand{\empy}[1]{{\color{darkorange}\emph{#1}}}
\newcommand{\empr}[1]{{\color{cardinalred}\emph{#1}}}
\newcommand{\examplebox}[2]{
\begin{tcolorbox}[colframe=darkcardinal,colback=boxgray,title=#1]
#2
\end{tcolorbox}}

\usetheme{Stanford} 
\input{./style_files_stanford/my_beamer_defs.sty}
\logo{\includegraphics[height=0.4in]{./style_files_stanford/SU_New_BlockStree_2color.png}}

% commands to relax beamer and subfig conflicts
% see here: https://tex.stackexchange.com/questions/426088/texlive-pretest-2018-beamer-and-subfig-collide
\makeatletter
\let\@@magyar@captionfix\relax
\makeatother

\title[Quantum error-correction]{Quantum error-correction}
%\subtitle{Subtitle Of Presentation}

%\beamertemplatenavigationsymbolsempty

% Allows using \insertframetitle on the footline
\providecommand\insertframetitle{}


\begin{document}

\author[Nicolás Tallar]{
	\begin{tabular}{c} 
	\Large
	%Author Name\\
    %\footnotesize \href{mailto:authoremail@stanford.edu}{authoremail@stanford.edu}
\end{tabular}
\vspace{-4ex}}

\institute{
	\vskip 5pt
	\begin{figure}[t]
		\centering
		\includegraphics[height=0.5in]{./images/logo_dc.jpg}
	\end{figure}
	\vskip 5pt
	Procesamiento cuántico de información\\
	Segundo cuatrimestre de 2020\\
	\vskip 3pt
}

% \date{June 15, 2020}
\date{\today}

\begin{noheadline}
\begin{frame}
    \frametitle{Introduction}
    \maketitle
\end{frame}
\end{noheadline}

\setbeamertemplate{itemize items}[default]
\setbeamertemplate{itemize subitem}[circle]

% Ideas de mejoras:
%  - Simplificar los estados para que se entiendan más
%  - Quitar un poco de info (capaz las diapos de análisis de que tan bien funciona?) como para poder hablar más tranquilo
%  - Ver si tengo algo donde pueda usar un lapiz sobre el pdf
%  - Capaz estaría bueno que las diapos estén más ordenadas para matchear lo que yo quiero decir así no me confunden
%  - Aclarar más el hecho que todo esto funciona solo para errores en 1 qubit

\begin{frame}
	\frametitle{Quantum error correction} % Table of contents slide, comment this block out to remove it
	\tableofcontents % Throughout your presentation, if you choose to use \section{} and \subsection{} commands, these will automatically be printed on this slide as an overview of your presentation
\end{frame}


\begin{frame}[allowframebreaks]
\frametitle{Intro}
    
    %%%%%%%%%%%%%%%%%%%%%%%%%%%%%%%%%%%%%%%%%%%%%%%%%%%%%%%%%%%%%%%%%%%%%%%%%%%%%%%%%%%%%%%%%%%%%%%%%%%%%%%%%%
    %
    %   Comenzar con lo del canal ruidoso, hablando del ejemplo del libro
    %   Continuar con como esto se puede después expandir a gates ruidosas
    %
    %%%%%%%%%%%%%%%%%%%%%%%%%%%%%%%%%%%%%%%%%%%%%%%%%%%%%%%%%%%%%%%%%%%%%%%%%%%%%%%%%%%%%%%%%%%%%%%%%%%%%%%%%%
    \textbf{Objetivo}:

    \begin{figure}[H]
        \centering
        \begin{quantikz}
            \lstick{$\ket{\psi}$} & \gate{Noisy \ channel} & \rstick{$\ket{\psi}$}\qw
        \end{quantikz}
    \end{figure}

    \vspace{0.6cm}

    \textit{Las ideas presentadas se usan como base para tolerar fallas en cualquier operación cuántica}

    \framebreak

    \textbf{Idea principal}: Agregar redundancia a los mensajes para que sean resilientes al ruido

    \vspace{0.3cm}

    \begin{figure}[H]
        \centering
        \begin{quantikz}
            \qw & \gate[3, nwires={2, 3}]{Encoding} & \qw & \gate[3]{Channel}   & \qw & \gate[3]{Decoding} & \qw  \\
                &                                   & \qw &                     & \qw &                    & \\
                &                                   & \qw &                     & \qw &                    &
        \end{quantikz}
    \end{figure}

    \vspace{0.3cm}

    \begin{itemize}
        \item Asumimos que el encodeo y desencodeo de mensajes se puede hacer sin error (existen formas de quitar esta asunci\'on)
    \end{itemize}
	
\end{frame}

\section{Corrección de errores clásicos}
\begin{frame}[allowframebreaks]
    \frametitle{Corrección de errores clásicos}

    \begin{figure}[H]
        \centering
        \begin{quantikz}
            \qw & \gate[3, nwires={2, 3}]{Encoding} & \qw & \gate[3]{Channel}   & \qw & \gate[3]{Decoding} & \qw  \\
                &                                   & \qw &                     & \qw &                    & \\
                &                                   & \qw &                     & \qw &                    &
        \end{quantikz}
    \end{figure}

    \begin{itemize}
        \item \textbf{Channel:} \textit{binary simmetric channel}, alterna cada bit con probabilidad \textit{p}
        \item \textbf{Encoding:}
        \begin{gather*}
            0 \rightarrow 000 \\
            1 \rightarrow 111
        \end{gather*}
        \item \textbf{Decoding:} Majority voting
    \end{itemize}

    \framebreak

    %%%%%%%%%%%%%%%%%%%%%%%%%%%%%%%%%%%%%%%%%%%%%%%%%%%%%%%%%%%%%%%%%%%%%%%%%%%%%%%%%%%%%%%%%%%%%%%%%%%%%%%%%%
    %
    %   Solo mencionar que haber corregido el error es mejor a no haberlo hecho,
    %   no entrar en detalles sobre porque
    %
    %%%%%%%%%%%%%%%%%%%%%%%%%%%%%%%%%%%%%%%%%%%%%%%%%%%%%%%%%%%%%%%%%%%%%%%%%%%%%%%%%%%%%%%%%%%%%%%%%%%%%%%%%%

    \begin{itemize}
        \item \textit{Sin corrección:} p
        \item \textit{Con corrección:} $p_e = 3p^2 - 2p^3$ \hfill $\leftarrow$ 2 o 3 bits se alternaron
        \begin{equation*}
            p_e < p\ si\ p < \frac{1}{2}
        \end{equation*}
        % = prob(2 flip) + prob(3 flip) = 3(1-p)p^2 + p^3
    \end{itemize}

\end{frame}


\section{Corrección de errores cuánticos}

\begin{frame}[allowframebreaks]
    \frametitle{Corrección de errores cuánticos}

    \begin{figure}[H]
        \centering
        \begin{quantikz}
            \qw & \gate[3, nwires={2, 3}]{Encoding} & \qw & \gate[3]{Channel}   & \qw & \gate[3]{Error \ detection} & \qw & \gate[3]{Recovery} & \qw  \\
                &                                   & \qw &                     & \qw &                             & \qw &                    & \\
                &                                   & \qw &                     & \qw &                             & \qw &                    & \\
                &                                   &     &                     &     & \cwbend{-3}                 & \cw & \cwbend{-3}        &
        \end{quantikz}
    \end{figure}

    \textbf{Dificultades:}
    \begin{itemize}
        \item No cloning
        \item Los errores son continuos
        \item Medir puede destruir información
    \end{itemize}

\end{frame}

\subsection{3 qubit flip code}

\begin{frame}[allowframebreaks]
    \frametitle{3 qubit flip code}

    \begin{figure}[H]
        \centering
        \begin{quantikz}
            \qw & \gate[3, nwires={2, 3}]{Encoding} & \qw & \gate[3]{Channel}   & \qw & \gate[3]{Error \ detection} & \qw & \gate[3]{Recovery} & \qw  \\
                &                                   & \qw &                     & \qw &                             & \qw &                    & \\
                &                                   & \qw &                     & \qw &                             & \qw &                    & \\
                &                                   &     &                     &     & \cwbend{-3}                 & \cw & \cwbend{-3}        &
        \end{quantikz}
    \end{figure}

    \begin{itemize}
        \item \textbf{Channel:} \textit{bit flip channel}, alterna $\ket{\psi}$ a $X \ket{\psi}$ con probabilidad \textit{p}
        \item \textbf{Encoding:} $a \ket{0} + b \ket{1} \rightarrow a \ket{000} + b \ket{111}$
        \begin{gather*}
            \ket{0} \rightarrow \ket{000} \\
            \ket{1} \rightarrow \ket{111}
        \end{gather*}
        \item \textbf{Error detection \& recovery:} error syndromes
    \end{itemize}

    \framebreak
    
    \begin{itemize}
        \item \textbf{Encoding:} $a \ket{0} + b \ket{1} \rightarrow a \ket{000} + b \ket{111}$
    \end{itemize}

    \vspace{0.3cm}

    \begin{figure}
        \centering
        \begin{subfigure}{0.2\textwidth}
            \centering
            \begin{gather*}
                \ket{0} \rightarrow \ket{000} \\
                \ket{1} \rightarrow \ket{111}
            \end{gather*}
        \end{subfigure}%
        \begin{subfigure}{0.5\textwidth}
            \centering
            \begin{quantikz}
                \lstick{$\ket{\psi}$} & \ctrl{1} & \ctrl{2} & \qw \\
                \lstick{$\ket{0}$}    & \targ{}  & \qw      & \qw \\
                \lstick{$\ket{0}$}    & \qw      & \targ{}  & \qw 
            \end{quantikz}
        \end{subfigure}
    \end{figure}

    \framebreak

    \begin{itemize}
        \item \textbf{Error detection \& recovery:}
        
        \textit{Error syndrome:} medición que no altera el estado medido
    \end{itemize}

    \begin{center}
        \begin{tabular}{ | l | l | }
            \hline
            Error detection & Recovery \\ 
            \hline
            $P_0 \equiv \ket{000}\bra{000} + \ket{111}\bra{111} \rightarrow$ no error       & $I \otimes I \otimes I$ \\
            $P_1 \equiv \ket{100}\bra{100} + \ket{011}\bra{011} \rightarrow$ first flipped  & $X \otimes I \otimes I$ \\
            $P_2 \equiv \ket{010}\bra{010} + \ket{101}\bra{101} \rightarrow$ second flipped & $I \otimes X \otimes I$ \\
            $P_3 \equiv \ket{001}\bra{001} + \ket{110}\bra{110} \rightarrow$ third flipped  & $I \otimes I \otimes X$ \\
            \hline
       \end{tabular}
    \end{center}
    % P0 + P1 + P2 + P3 = I -> la medición me va a dar algunos de ellos
    % Ante un caso (no error/bit i flippado) Pi da con probabilidad 1 y no altera el estado

    
\end{frame}

\begin{frame}[allowframebreaks]

    %%%%%%%%%%%%%%%%%%%%%%%%%%%%%%%%%%%%%%%%%%%%%%%%%%%%%%%%%%%%%%%%%%%%%%%%%%%%%%%%%%%%%%%%%%%%%%%%%%%%%%%%%%
    %
    %   Solo mencionar que haber corregido el error es mejor a no haberlo hecho,
    %   ???
    %
    %%%%%%%%%%%%%%%%%%%%%%%%%%%%%%%%%%%%%%%%%%%%%%%%%%%%%%%%%%%%%%%%%%%%%%%%%%%%%%%%%%%%%%%%%%%%%%%%%%%%%%%%%%

    \frametitle{3 qubit flip code - análisis de error}

    \textbf{Análisis simplificado}: El algoritmo funciona perfectamente si hay bit flips en 1 o menos de los 3 qubits, lo cual sucede con probabilidad
    \[
        p_c = (1 - p)^3 + 3p(1 - p)^2 = 1 - 3p^2 + 2p^3
    \]

    Comparado con no utilizarlo:
    \[
        p_c > 1 - p \ si \ p < \frac{1}{2}
    \]

    \framebreak

    \textbf{Fidelidad:} medición de cercanía de estados
    \begin{gather*}
        F(\ket{\psi}, \rho) = \sqrt{\bra{\psi} \rho \ket{\psi}} \\
        0 \leqslant F(\ket{\psi}, \rho) \leqslant 1
    \end{gather*}

    \textbf{Fidelidad sin corrección:}
    \begin{gather*}
        \rho = (1 - p) \ket{\psi}\bra{\psi} + p X \ket{\psi}\bra{\psi} X \\
        F(\ket{\psi}, \rho) = \sqrt{1 - p}
    \end{gather*}
    
    \textbf{Fidelidad mínima con corrección:}
    \begin{gather*}
        \rho = [(1 - p)^3 + 3p(1 - p)^2] \ket{\psi}\bra{\psi} + ... \\
        F(\ket{\psi}, \rho) \geqslant \sqrt{(1 - p)^3 + 3p(1 - p)^2}
    \end{gather*}

\end{frame}

\begin{frame}[allowframebreaks]
    \frametitle{3 qubit flip code - alternative error detection}

    En vez de medir los projectores, se miden los observables
    \begin{itemize}
        \item $Z_1 Z_2 = Z \otimes Z \otimes I = (\ket{00}\bra{00}+\ket{11}\bra{11}) \otimes I - (\ket{01}\bra{01}+\ket{10}\bra{10}) \otimes I$ \\
        \hspace{0.7cm} $\rightarrow$ +1, $1^{er}$ y $2^{do}$ qubit son iguales, o -1 si son distintos
        \item $Z_2 Z_3 = I \otimes Z \otimes Z = I \otimes (\ket{00}\bra{00}+\ket{11}\bra{11}) - I \otimes (\ket{01}\bra{01}+\ket{10}\bra{10})$ \\
        \hspace{0.7cm} $\rightarrow$ +1, $2^{do}$ y $3^{er}$ qubit son iguales, o -1 si son distintos
    \end{itemize}

    \begin{center}
        \begin{tabular}{ | l | l | l | }
            \hline
            Projectores & $1^{er}$ medición & $2^{da}$ medición \\ 
            \hline
            $P_0$ & +1 & +1 \\
            $P_1$ & -1 & +1 \\
            $P_2$ & -1 & -1 \\
            $P_3$ & +1 & -1 \\
            \hline
       \end{tabular}
    \end{center}

\end{frame}

\subsection{3 qubit phase code}

\begin{frame}[allowframebreaks]
    \frametitle{3 qubit phase code}

    \begin{itemize}
        \item \textbf{Channel:} \textit{phase flip channel}, alterna $\ket{\psi}$ a $Z \ket{\psi}$ con probabilidad \textit{p} \\
        \vspace{0.6cm}
        Es \textit{unitarly equivalent} a los \textit{bit flip channel} ya que $X = HZH$:

        \begin{figure}[H]
            \centering
            \begin{quantikz}
                \qw & \gate{Bit \ flip \ channel} & \qw
            \end{quantikz} \\
            = \\
            \begin{quantikz}
                \qw & \gate{H} & \gate{Phase \ flip \ channel} & \gate{H^{\dag} = H} & \qw
            \end{quantikz}
        \end{figure}

    \end{itemize}

    \framebreak

    \begin{figure}[H]
        \centering
        \begin{quantikz}
            \qw & \gate[3, nwires={2, 3}]{Encoding} & \qw & \gate[3]{Channel}   & \qw & \gate[3]{Error \ detection} & \qw & \gate[3]{Recovery} & \qw  \\
                &                                   & \qw &                     & \qw &                             & \qw &                    & \\
                &                                   & \qw &                     & \qw &                             & \qw &                    & \\
                &                                   &     &                     &     & \cwbend{-3}                 & \cw & \cwbend{-3}        &
        \end{quantikz}
    \end{figure}

    \begin{itemize}
        \item \textbf{Channel:} \textit{phase flip channel}, alterna $\ket{\psi}$ a $Z \ket{\psi}$ con probabilidad \textit{p}
        \item \textbf{Encoding:} $a \ket{0} + b \ket{1} \rightarrow a \ket{+++} + b \ket{---}$
        \begin{gather*}
            \ket{0} \rightarrow \ket{+++} \\
            \ket{1} \rightarrow \ket{---}
        \end{gather*}
        \item \textbf{Error detection \& recovery:} error syndromes
    \end{itemize}

    \framebreak
    
    \begin{itemize}
        \item \textbf{Encoding:} $a \ket{0} + b \ket{1} \rightarrow a \ket{+++} + b \ket{---}$
    \end{itemize}

    \begin{figure}
        \centering
        \begin{subfigure}{0.2\textwidth}
            \centering
            \begin{gather*}
                \ket{0} \rightarrow \ket{+++} \\
                \ket{1} \rightarrow \ket{---}
            \end{gather*}
        \end{subfigure}%
        \begin{subfigure}{0.5\textwidth}
            \centering
            \begin{tikzpicture}
                \node[scale=0.8] {
                    \begin{quantikz}
                        \lstick{$\ket{\psi}$} & \ctrl{1} & \ctrl{2} & \gate{H} & \qw \\
                        \lstick{$\ket{0}$}    & \targ{}  & \qw      & \gate{H} & \qw \\
                        \lstick{$\ket{0}$}    & \qw      & \targ{}  & \gate{H} & \qw 
                    \end{quantikz}
                };
            \end{tikzpicture}
        \end{subfigure}
    \end{figure}


    \begin{itemize}
        \item \textbf{Error detection \& recovery:}
    \end{itemize}

    \begin{figure}
        \centering
        \begin{subfigure}{0.2\textwidth}
            \[
                P_j \rightarrow H^{\otimes^3} P_j H^{\otimes^3}
            \]
            \subcaption{Projectores adaptados}
            % Las Hs del lado izquierdo son para que el estado no cambie
        \end{subfigure}%
        \begin{subfigure}{0.5\textwidth}
            \centering
            \begin{gather*}
                H^{\otimes^3} Z_1 Z_2 H^{\otimes^3} = X_1 X_2 \\
                H^{\otimes^3} Z_2 Z_3 H^{\otimes^3} = X_2 X_3
            \end{gather*}
            \subcaption{Observables adaptados}
            % Las Hs del lado izquierdo son para que el estado no cambie
            % X = |+><+| - |-><-|
        \end{subfigure}
    \end{figure}

    \begin{itemize}
        \item \textbf{Análisis de error:} mismos resultados que los de \textit{3 qubit flip code}
    \end{itemize}

\end{frame}


\subsection{Shor code}

\begin{frame}[allowframebreaks]
    \frametitle{Shor code}

    \begin{figure}[H]
        \centering
        \begin{quantikz}
            \qw & \gate[3, nwires={2, 3}]{Encoding} & \qw & \gate[3]{Channel}   & \qw & \gate[3]{Error \ detection} & \qw & \gate[3]{Recovery} & \qw  \\
                &                                   & \qw &                     & \qw &                             & \qw &                    & \\
                &                                   & \qw &                     & \qw &                             & \qw &                    & \\
                &                                   &     &                     &     & \cwbend{-3}                 & \cw & \cwbend{-3}        &
        \end{quantikz}
    \end{figure}

    \begin{itemize}
        \item \textbf{Channel:} cualquier error en un único qubit
        \item \textbf{Encoding:} concatenation de 3 qubit phase flip code y 3 qubit bit flip code 
        \item \textbf{Error detection \& recovery:} primero corriendo el procedimiento para bit flip y luego para phase flip
    \end{itemize}

    \framebreak
    
    \begin{itemize}
        \item \textbf{Encoding:} concatenation de 3 qubit phase flip code y 3 qubit bit flip code 
    \end{itemize}

    \begin{figure}
        \centering
        \begin{subfigure}{0.5\textwidth}
            \begin{gather*}
                \ket{0} \rightarrow \ket{0_L} \equiv \frac{(\ket{000} + \ket{111})^{\otimes^3}}{2\sqrt{2}} \\
                \ket{1} \rightarrow \ket{1_L} \equiv \frac{(\ket{000} - \ket{111})^{\otimes^3}}{2\sqrt{2}}
            \end{gather*}
        \end{subfigure}%
        \begin{subfigure}{0.5\textwidth}
            \centering
        \begin{tikzpicture}
        \node[scale=0.75] {
        \begin{quantikz}
            \lstick{$\ket{\psi}$} & \ctrl{3} & \ctrl{6} & \gate{H}              & \qw                 & \ctrl{1} & \ctrl{2} & \qw \\
                                  &          &          &                       & \lstick{$\ket{0}$}  & \targ{}  & \qw      & \qw \\
                                  &          &          &                       & \lstick{$\ket{0}$}  & \qw      & \targ{}  & \qw \\
            \lstick{$\ket{0}$}    & \targ{}  & \qw      & \gate{H}              & \qw                 & \ctrl{1} & \ctrl{2} & \qw \\
                                  &          &          &                       & \lstick{$\ket{0}$}  & \targ{}  & \qw      & \qw \\
                                  &          &          &                       & \lstick{$\ket{0}$}  & \qw      & \targ{}  & \qw \\
            \lstick{$\ket{0}$}    & \qw      & \targ{}  & \gate{H}              & \qw                 & \ctrl{1} & \ctrl{2} & \qw \\
                                  &          &          &                       & \lstick{$\ket{0}$}  & \targ{}  & \qw      & \qw \\
                                  &          &          &                       & \lstick{$\ket{0}$}  & \qw      & \targ{}  & \qw
        \end{quantikz}
        };
        \end{tikzpicture}
        \end{subfigure}
    \end{figure}

    \framebreak

    \begin{itemize}
        \item \textbf{Error detection \& recovery:} suponiendo que hubo bit flip en algún qubit:
    
        \vspace{0.4cm}

        \begin{itemize}
            \item \textit{Detection:} Son detectadas mediante mediciones de $Z_iZ_j$
            \item \textit{Recovery:} Se aplica X al qubit afectado
        \end{itemize}
    \end{itemize}

    \framebreak

    \begin{itemize}
        \item \textbf{Error detection \& recovery:} suponiendo que hubo phase flip en algún qubit:
    
        \vspace{0.4cm}

        \begin{itemize}
            \item \textit{Detection:} Phase flip en un qubit causa que se cambie el signo del bloque afectado
            
            $\ket{000} + \ket{111} \rightarrow \ket{000} - \ket{111}$

            Por lo que puede ser detectado comparando los signos de los bloques % FIXME: como se hace esto?

            \item \textit{Recovery:} Se aplica Z a cualquier qubit del bloque afectado
        \end{itemize}
    \end{itemize}

    \framebreak

    \begin{itemize}
        \item \textbf{Error detection \& recovery:} suponiendo que hubo phase y bit flip en algún qubit:
    
        \vspace{0.4cm}

        \begin{enumerate}
            \item Correr el procedimiento de corrección y detección de bit flip, después del cual solo quedará un phase flip en el qubit afectado
            \item Correr el procedimiento de corrección y detección de phase flip
        \end{enumerate}
    \end{itemize}

    \framebreak

    \begin{itemize}
        \item \textbf{Error detection \& recovery:} cualquier error en algún qubit:
    
        \vspace{0.2cm}

        Representando el error como una operación que preserva traza $\xi$, dado un estado inicial $\ket{\psi}$:
        \[
            \xi(\ket{\psi}\bra{\psi}) = \sum_i{E_i\ket{\psi}\bra{\psi}E_i^{\dag}}
        \]

        Cada $E_i$ puede expandirse como:
        \[
            E_i = e_{i0}I + e_{i1}X_1 + e_{i2}Z_1 + e_{i3}X_1Z_1
        \]
        
        \framebreak

        Entonces un error arbitrario se puede pensar como que lleva un estado $\psi$ a una superposición de los estados:
        \[
            \ket{\psi}, X_1\ket{\psi}, Z_1\ket{\psi}, X_1Z_1\ket{\psi}
        \]
        % En realidad también estará en estados mixtos como X1|psi><psi|Z1, que no sobrevivirán a los procedimientos

        Por lo que el procedimiento para corregirlo es:

        \begin{enumerate}
            \item Correr el procedimiento de corrección y detección de bit flip, después del cual el estado quedará en superposición de:
            
            \[
                \ket{\psi}, X_1\ket{\psi}
            \]

            \item Correr el procedimiento de corrección y detección de phase flip
        \end{enumerate}
    \end{itemize}
\end{frame}

\begin{frame}[allowframebreaks]
    \frametitle{Shor code - error en más de un qubit}

    Que pasa si el error afecta a más de un qubit?

    \begin{itemize}
        \item Si el error de cada qubit es independiente se lo puede achicar mediante los procedimientos descritos hasta ahora
        \item Sino hay que utilizar otros procedimientos
    \end{itemize}

\end{frame}

\begin{frame}[allowframebreaks]
    \frametitle{Bibliography}

    \nocite{*}
    \printbibliography
\end{frame}

\end{document}